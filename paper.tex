
\documentclass{article}

\usepackage{natbib}
\usepackage{times}

% hyperref must be last package.
\usepackage[colorlinks=true,citecolor=blue,hyperfootnotes=false]{hyperref}

\title{How Things Really Are\\{\Large Actually and Potentially}}
\author{Thomas E. Vaughan}

\begin{document}

\maketitle

\section{Introduction}

There is a great war raging.  It is a cultural war between Western civilization
and her enemies.  By ``Western Civilization'' I mean what eventually emerged
from the confluence of ideas in ancient Israel and classical Greece.  First,
these ideas became intertwined in the writing of the Septuagint.  Then they
were united in the minds of Christians, who converted the declining Roman
empire to the Faith, defended Europe against the onslaught of Muslim invaders,
and finally established a new civilization, among whose fruits are modern
science\footnote{%
   The terms ``science'' and ``scientific'' are by etymology misleading.  In
   Latin, the verb ``scire'' means ``to know,'' but what we call a ``scientific
   theory,'' is in many cases something that can never be known as a truth.
   There are, of course, truths in modern science; primarily, these are
   \emph{observational} truths: Every repeatable observation is a truth of
   modern science.  Further, a scientific theory that refers only to
   perceptible things might in principle be a truth, if it be a descriptive
   theory, such as the arrangement of the bones in the human body.  However, a
   scientific theory that refers to something imperceptible, like the electron
   for example, might some day be superseded by a new theory making no
   reference to the imperceptible thing.  Its existence in reality is not
   certain, and so the theory referring to it cannot be known as a truth.%
}
and laws respecting the rights of man.  Western civilization has many enemies.
For example, Islam in its main Sunni and Shiite forms remains opposed to
Western civilization.\footnote{%
   At least one ideas at the root of Western civilization is opposed to a
   fundamental teaching of Islam.  The Western idea that every human being is
   made in the image of God is the foundation of universal human rights.
   However, the idea that man is made in the image of God is blasphemy under
   Islam.
   
   Although Aristotle became known to the West because early Islamic
   philosophers had studied and preserved his work, this is ironic.  Just as
   Aristotle was being incorporated into Christian thought, Aristotle was being
   purged from Islamic thought. The Islamic tradition, which stresses the power
   of God, sees Aristotle's view of nature as competitive with the idea of
   God's omnipotence.  Islamic thought became dominated by occasionalism after
   about the year 1000~AD.
}
However, a particular, modern form of atheism also presents an existential
challenge to the West.  The New Atheists\footnote{%
   ``New Atheism'' is a term widely used to refer to ideas popularized by
   Richard Dawkins, Sam Harris, Daniel Dennett, and Christopher Hitchens in the
   first decade of the 21st Centiry.
}
deny the Aristotelian metaphysical basis for understanding reality while
claiming for themselves, much as the Soviet Communists did during the Cold War,
the mantle of science.  The present talk is motivated by my interest in the
attack on the Aristotelian philosophical foundation of Western Civilization
ironically by the those who imagine themselves to be advocates of modern
science.  I point to a philosophical front in the war: the conflict over how
things really are.

The story of the conflict begins in ancient Israel.  The relevant idea from
{\it Genesis} is that things have not always existed.  God, Who is not a thing,
initially created things from nothing.  A new thing, such as a pot, might come
into existence by transformation from a pre-existing thing, such as a lump of
clay.  However, according God's revelation, there was in the past a beginning
when the first things were created ex nihilo, not by transformation.  The
fundamental distinction made by the ancient Israelites is between God and what
God creates.

Another thread of the story begins in classical Greece.  The early philosophers
struggled to understand how things change, and their tradition culminated in
the writing of Aristotle.  Aristotle explained that for a thing to change is
for what existed only potentially in the thing to become actual.  The
fundamental distinction is between actuality and potentiality.  This
distinction is the root of what has come to be called ``Aristotelian
metaphysics.''\footnote{%
   The word, ``metaphysics,'' originates from the accidental fact that in the
   traditional list of Aristotle's writings, his writing on the distinction
   between act and potency came after (meta) his writing on nature (physics).
   These days, any theory about what lies at the root of being or becoming is
   called ``a metaphysical theory.''%
}

When, after having been lost for a time, Aristotle was reintroduced into the
West by Saint Thomas Aquinas, he perfected the metaphysical theory so that it
could explain not only change by transformation but also creation ex nihilo.
Aquinas showed how, even if there had been no beginning in time, there would
still be creation ex nihilo at every moment in time!  Although Aristotle's
metaphysical theory naturally led to the emergence of modern science in the
West, Aristotle's opponents in the present conflict ironically mount their
attack ostensibly under the banner of science.  The self-proclaimed ``science
advocate,'' opposing the Aristotelian position, typically proposes two main
errors:
\begin{enumerate}
   \item \emph{scientism}, according to which everything that exists is
      describable by modern science, and
   \item \emph{scientific realism}, according to which an imperceptible thing
      proposed as part of a scientific theory is certain to exist.
\end{enumerate}
We shall see in what follows how Aristotelian metaphysics provides powerful
arguments against these errors.  First, however, we must explore the key
Aristotelian distinction, which allowed Aquinas to form a synthesis of the
Judeo-Christian idea of creation and the Greek explanation of change.

\section{Act and Potency}

\subsection{The General Theory}

\subsubsection{Example: Silly Putty}

\subsection{Causal Powers and Laws of Nature}

\section{Efficient and Final Causality}

\section{Formal and Material Causality}

\section{Essence and Existence}

\bibliographystyle{plainnat}

\begin{thebibliography}{}

\bibitem[Feser(2014)Feser]{sm}
Feser, E.\ \ {\it Scholastic Metaphysics: A Contemporary Introduction.}
      editiones scholasticae (www.editiones-scholasticae.de). Heusenstamm.
      2014.

\end{thebibliography}

\end{document}

