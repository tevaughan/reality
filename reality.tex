
\newcommand{\doctitle}{How Things Really Are, Actually and Potentially}

\documentclass[twocolumn]{article}

\usepackage{fancyhdr}
\usepackage{ftnright}
\usepackage{graphicx}
\usepackage{lastpage}
\usepackage{natbib}
\usepackage{paralist} % for compactitem
\usepackage{times}
\usepackage{vmargin}

% hyperref must be last package.
\usepackage[colorlinks=true,citecolor=blue,hyperfootnotes=false]{hyperref}

% vmargin setup
\setpapersize{USletter}
\setmarginsrb%
{0.4in}%           left
{0.4in}%           top
{0.4in}%           right
{0.5in}%           bottom
{2\baselineskip}%  headheight
{2\baselineskip}%  headsep
{3\baselineskip}%  footheight
{4\baselineskip}%  footskip

% mydate macro
\newcommand{\mydate}{%
   \number\year\space%
   \ifcase\month\or%
      Jan\or\ Feb\or\ Mar\or\ Apr\or\ May\or\ Jun\or%
      Jul\or\ Aug\or\ Sep\or\ Oct\or\ Nov\or\ Dec
   \fi\space%
   \number\day%
}

% fancyhdr settings
\pagestyle{fancy}
\lhead{\sffamily\textbf{\doctitle}}
\chead{}
\rhead{\sffamily \thepage~of~\pageref{LastPage}}
\renewcommand{\headrulewidth}{1pt}
\renewcommand{\footrulewidth}{1pt}
\lfoot{%
   \scriptsize\sffamily
   \begin{minipage}{0.95\textwidth}
   Copyright~\copyright~2017~Thomas E. Vaughan.\ \ \
   PDF image generated on \mydate.\ \ \
   Source code and on-line PDF version at
      \url{https://github.com/tevaughan/reality}.\ \ \
   Permission is granted to copy, distribute and/or modify this document under
   the terms of the GNU Free Documentation License, Version 1.3 or any later
   version published by the Free Software Foundation; with no Invariant
   Sections, no Front-Cover Texts, and no Back-Cover Texts.  A copy of the
   license is included in the section entitled ``GNU Free Documentation
   License''.
   \end{minipage}%
}
\cfoot{}
\rfoot{%
   \begin{minipage}{0.05\textwidth}
   \begin{flushright}
   \includegraphics[width=0.85\textwidth]{logo}
   \end{flushright}
   \end{minipage}%
}

\renewcommand{\footnoterule}{%
   \kern -3pt
   \hrule width 0.5 \columnwidth
   \kern 2.6pt
}

\begin{document}

\thispagestyle{fancy}

\section{Introduction}

For hundreds of years, a great war of ideas has been raging.  In the early 17th
Century, just as Western civilization was beginning to use modern science for
rapid improvement in technology, she was already suffering a deep corruption.
The long battle to subdue her heart was about to begin.  We are even today
fighting to defend her from her enemies.

The West has a philosophical foundation.  It was laid in the 13th Century, when
St.~Thomas of Aquinas finished the metaphysical\footnote{%
   According to \cite{vis2017}, the word, ``metaphysics,'' originates from the
   order of the books in the traditional list of Aristotle's writings.  His
   writing on first philosophy, first science, wisdom, and theology have come
   down to us in 14 books.  In the traditional ordering of Aristotle's books,
   these came after (meta) his books on nature (physics).  These days, any
   theory about what lies at the root of being or of change is called ``a
   metaphysical theory.''%
}
project initiated by the classical Greeks.  By 1600, though, too few of the
West's brightest students had kept philosophical thought firmly connected to
right principles.  The decay began in earnest with Ren\'e Descartes.  Although
his mathematical innovations greatly benefited technology, his philosophical
innovations were corruptions.  Their popularity led to further corruptions and
eventually to open war in the West against her own heart.\footnote{%
   \citet[Chapter IV]{b1954} describes how Descartes divided creation into
   extended things (r\={e}s extens\={a}s) and thinking things (r\={e}s
   c\={o}gitant\={e}s). He imagined every material thing to be defined entirely
   by its geometric extension in three-dimensional space, and he imagined every
   thinking thing to have no extension whatsoever.  By denying the reality of
   whatever is not geometric in a physical thing, Descartes initiated the
   return of the Eleatic insistence on the essentially illusory character of
   sense experience.  Descartes admitted that macroscopic geometry is truly
   sensed and, unlike the Eleatics, he admitted that local motion is truly
   sensed.  However, like the Eleatics, he denied most of what the senses
   report. The sensation of color, heat, cold, sound, taste, and smell were
   regarded by Descartes as illusions.  For him, any such a sensation is due
   only to the shapes and the motions of microscopic things and makes no direct
   report of any quality existing in the object of sensation.  This gives rise
   to the problem of qualia, and his dichotomy between thinking and extended
   things gives rise to the mind-body problem; neither of these problems need
   exist with a proper metaphysics.
}

\subsection{The Origins of Western Civilization}

By ``Western civilization'' I mean the civilization that grew out of the
marriage between ancient Israel and classical Greece.  Like any living thing in
the world, the West has struggled to survive since the moment of her birth, but
her difficulties are multiplied because of who she is.  The Western tradition
includes bold claims about the dignity of man and about what is objectively
real.  The West has not only enemies who might naturally oppose any
civilization but also enemies who dispute her challenging claims.  In this
talk, I shall focus on the Western view of reality---in light of Aquinas---and
on a particular enemy.

\subsubsection{Overview}

Let us reflect on some points of the West's development.  First, the Hebrew and
the Greek traditions became intertwined in the writing of the {\it
Septuagint}.\footnote{%
   The {\it Septuagint} is the first translation of the {\it Tanakh} (what
   Christians call the {\it Old Testament}).  The translation, into Greek, was
   made before the Christian era.  \cite{vdh1912} provides a detailed overview.
   \cite{r2005, r2007} comments on the marriage of the Greek and the Hebrew
   cultures in the writing of the {\it Septuagint}. (I do not have page
   references because I do not have my copies of those books as I write the
   present talk.)  In at least one of {\it Truth and Tolerance} and {\it Jesus
   of Nazareth}, Ratzinger writes of the union of the cultures as a marriage,
   of which the {\it Septuagint} is the offspring. Further, Ratzinger indicates
   that the relationship between God and man is completed, in a sense, by this
   marriage, which matures in Christianity.  The Hebrew culture results from
   God's reaching down to man and man's faithful response; the Greek culture
   results from man's reaching up toward God through the intellect.
}
The two became one in Christianity, which requires both traditions in order to
express doctrine.  Christians converted the declining Roman empire to the
Faith, defended Europe against the onslaught of Muslim invaders,\footnote{%
   \citet[Chapter 4]{b1938} gives a concise overview of the circumstances
   leading up to the First Crusade.
}
and finally established a new civilization, among whose fruits are modern
science\footnote{%
   The terms ``science'' and ``scientific,'' by etymology, could be misleading.
   In Latin, the verb ``scire'' means ``to know,'' but many a scientific theory
   cannot be known as a truth.  Modern science does have truths: Every
   repeatable perception resulting from a carefully described experiment or
   observation is a truth that has the power to prove a theory false, even a
   theory that had long been treated as standard.  However, a scientific theory
   that refers to something imperceptible, like the electron, might some day be
   superseded by a new theory making no reference to the imperceptible thing.
   The imperceptible thing was proposed as a possible reason for the
   perceptible result of an experiment.  The imperceptible thing's existence is
   not certain, and so the theory referring to it cannot be known as a truth.
   There are many who would claim that the electron's existence is certain, but
   such ``entity realism'' has problems.  \cite{c2016} gives an overview of
   scientific realism.%
}
and laws respecting the rights of man.  Western civilization is not identical
with Christian civilization; for example, one might participate fully both in
Western tradition and in Jewish tradition.\footnote{%
   Although Jewish tradition is consistent with Western civilization, Islamic
   tradition appears not to be.  A person who calls himself ``Muslim'' might
   participate fully in Western culture.  However, if his support for universal
   human rights be grounded in the belief that man is created in the image of
   God, then he participates \emph{against his own religion}, which condemns
   the Judeo-Christian idea of man.  Even if he find some other basis for
   universal human rights, he will have trouble reconciling it with Islamic
   tradition.  \cite{as2003} point out that both the Sunni and the Shiite
   traditions in Islam are opposed to universal human rights (unless the only
   humans in existence were Muslims).
}
Still, her existence is principally due to Christianity.

\subsubsection{Ancient Israel and Creation}

The story of the West begins in ancient Israel.  The fundamental idea from {\it
Genesis}~1 is that things have not always existed.  God, Who is not a thing,
initially created things.

A new thing, such as a pot, might come into existence by transformation from a
pre-existing thing, such as a lump of clay.  In the myths of the peoples
surrounding the ancient Israelites, the universe had no ultimate beginning.
The pagan myth of origin, regardless of its details, describes the origin of
ordinary things as transformed from something that had existed before.

In contrast, according God's revelation to the Israelites, there was a
beginning when the first things were created from nothing (ex nihilo), not by
transformation.  The fundamental distinction made by the ancient Israelites is
between God and what God creates.

\subsubsection{Classical Greece and Transformation}

Another thread of the story begins in classical Greece.  The early philosophers
struggled to understand the apparent conflict between how a thing remains the
same and how it changes.  The difficulty of understanding the apparent
simultaneity of permanence and change, of being and becoming, led some to
conclude that change is not real; others concluded that there is no real
permanence.  On each side, sense experience was viewed as illusory in some way.
The Greek philosophical tradition culminated, however, in the writing of
Aristotle, who stood up for the reality of sense experience and for the mind's
ability to abstract from it.

Aristotle proposed that there are, fundamentally, two different kinds of
reality: actuality and potentiality.  By explaining that
\begin{quote}
   in any change, what had already existed as a potentiality becomes an
   actuality,
\end{quote}
he showed how both permanence (in what remains actual) and change (in what
becomes actual) are real. Neither the sense experience of change nor the sense
experience of permanence is an illusion.  The distinction between actuality and
potentiality is the root of what is called ``Aristotelian metaphysics,'' and it
led to modern science, which takes sense experiences as the fundamental data in
need of theoretical explanation.

\subsubsection{Synthesis in Christianity}

Working at first with translations into Latin from Arabic but eventually with
translations from the original Greek,\footnote{%
   After the fall of the Roman Empire, copies of Aristotle were not maintained
   in the West.  By 500~AD, the knowledge of Greek and the copying of Greek
   manuscripts had almost completely disappeared from the West.  Some of the
   Greek classics had been translated into Latin and survived in the West, but
   most of Aristotle's works were lost to the West.  As Spain was conquered by
   Christians in the 1100s, after about 400 years of Islamic rule, some Arabic
   copies of Aristotle became available.  A hundred years later, after the
   Fourth Crusade and the sacking of Constantinople by the crusaders, original
   Greek manuscripts of Aristotle became available in the West.%
}
Aquinas perfected the metaphysical theory so that it could explain not only
change by transformation but also creation ex nihilo.  Using Aristotle's
metaphysical principles of actuality and potentiality, Aquinas defined creation
ex nihilo as the union of an essence, which potentially exists, with the act of
existence.  Then he showed how, even if there had been no beginning in time,
there would still be creation ex nihilo at every moment in time.\footnote{%
   Aquinas shows how we cannot know by observation whether the universe have a
   finite age.  Only revelation can give us certainty that time began.  (Even a
   useful theory like the Big Bang cannot give us certainty that time began.)
   However, Aquinas also shows that God creates the universe ex nihilo at every
   moment in time.  So creation ex nihilo does not require a beginning in time.
   That is, God could create the universe such that it always existed, if God
   wanted to.  See the translation of the {\it Summa Theologiae} by the
   \citet[I, Q45, A1, and I, Q46, A1]{e1920}.
}

Even before Aquinas, there was in Christianity a deep union between the Greek
philosophical tradition and the Hebrew tradition.  Although a Christian who
does not follow Aquinas may still be fully a Christian---just as a Westerner
who is not a Christian may still participate fully in Western
civilization---Aquinas nevertheless represents the union of the fullness of the
Greek intellectual tradition and the Hebrew tradition preserved in
Christianity.  In this sense, Aquinas lies at the heart of Western
civilization.

\subsection{A Recent Enemy}

The West has many enemies, both within countries formed by her ideals and
without.  Since the time of Descartes, who denied Aristotelian metaphysics, the
West has had philosophical enemies within her own ranks.  This continues down
to the present day.

A recent enemy of Western civilization is a form of atheism.  The New
Atheists,\footnote{%
   ``New Atheism'' is a term widely used to refer to ideas popularized by
   Richard Dawkins, Sam Harris, Daniel Dennett, and Christopher Hitchens in the
   first decade of the 21st Century.  \cite{f2008} devotes a whole book to the
   refutation of the philosophical errors of the New Atheists.
}
like Richard Dawkins, aggressively dismiss the Aristotelian metaphysical basis
for understanding reality.\footnote{%
   \citet[Chapter 3]{d2006} tries to refute the five arguments famously given
   by Aquinas.  In the attempted refutation, Dawkins at least implicitly denies
   the key metaphysical principles of Aristotle.  Apparently unaware of what he
   is doing, Dawkins fails to say what his own basic metaphysical assumptions
   are.
}
Although Aristotle's metaphysical theory naturally led to the emergence of
modern science in the West,\footnote{%
   \citet[Page 36]{f2014} points out that Aristotle's middle ground between the
   apparent extremes of the Eleatics and Heraclitus is precisely what is needed
   for something like modern science to work.  Aristotle more generally
   \citep[Pages 164--171]{f2014} developed the idea of a natural substance, and
   his basic thesis was that sense observations can be used to learn about
   nature.  The incorporation of these ideas into Christianity through Aquinas
   and the Catholic Church's systematic investigation of miraculous claims, to
   see if each claim had a natural explanation, laid the groundwork for the
   emergence of modern science in the West.
}
Aristotle's opponents today ironically mount their attack under the banner of
science.  The self-proclaimed advocate\footnote{%
   Bill Nye, for example, is one of the most famous advocates of science.  Yet
   in advocating not merely science but scientism, he makes a philosophical
   error.  Bishop Robert Barron appears in a short movie in which he talks
   about Bill Nye's scientism:
   \url{https://www.youtube.com/watch?v=SH_Njsa0zVQ}.%
}
of science typically holds an erroneous view, \emph{scientism},\footnote{%
   Arguably the initial, definitive use of the word, ``scientism,'' is given by
   \cite{s1991}.  My definition is essentially the one given by \cite{h2011}.
   There are many arguments against scientism.  See, for example, Thomas
   Nagel's argument from the problem of qualia, or subjective experience.
   Nagel's view is summarized here:
   \url{http://www.nybooks.com/articles/2017/06/08/how-to-imagine-consciousness},
   in his response to a comment by a professor Black.  Note that Nagel is an
   atheist, though he is out of favor with the typical modern atheist, whose
   materialism takes the form of scientism.  \cite{f2014}, in the introductory
   chapter of his book, summarizes all of the main arguments against
   scientism.%
}
according to which everything that exists is describable by modern science.

A key problem with scientism is that modern science is not itself describable
by modern science.  However, according to the proper metaphysical view, we
should expect modern science to provide good explanations of how physical
things behave.\footnote{%
   \cite{f2012} points out how the promoter of scientism is driven to
   adopt unexamined metaphysical assumptions.
   \begin{quote}
      [T]he distinction between actuality and potentiality, the principle of
      causality, and other fundamental elements of the Aristotelian conception
      of nature are among the preconditions of any possible material world
      susceptible of scientific study.

      That is why no findings of empirical science can undermine the claims of
      metaphysics and the philosophy of nature.  It is also why no findings of
      empirical science can undermine the Aristotelian-Thomistic arguments for
      the existence of God, for these are grounded in premises drawn, not from
      natural science, but from metaphysics and the philosophy of nature.  Now
      that does not mean that these arguments of natural theology are not
      susceptible of rational evaluation and criticism.  What it means is that
      such evaluation and criticism will have to be philosophical and
      metaphysical, rather than empirical, in nature.  Nor is natural theology
      in this regard at all different from atheism.  Atheists who think they
      are arguing from \emph{purely scientific} premises never really are.
      They are, without exception, arguing from metaphysical assumptions---and
      usually unexamined ones at that---that are first read into empirical
      science and then read back out, like the rabbit the magician can pull out
      of the hat only because he's first hidden it there.
   \end{quote}%
}
The first step in exploring proper metaphysics is to explore the key
Aristotelian distinction.

\section{The Actual and the Potential}

Aristotle distinguished between real, actual existence and real, potential
existence in order to explain both the reality of change and the reality of
permanence.

\subsection{The Eleatics}

In the Greek colony of Elea in Southern Italy---and before the time of
Socrates---Parmenides and Zeno\footnote{%
   Zeno wrote many paradoxes, but what is commonly presented as ``Zeno's
   paradox'' is that one cannot travel from point~A to point~B along a straight
   line.  First one would have to reach the midpoint~C between A~and~B, but,
   before that, one would have to reach the midpoint between A~and~C, etc.  The
   infinite regression was intended to show the absurdity of local motion.
   Zeno, like Parmenides, appeared to deny both the reality of all change and
   the multiplicity of beings.%
}
appear to have held that change is unreal.  In any change something new arises;
what is new did not exist before the change.  Denying that new existence can
arise from non-existence, the Eleatics, at least on one interpretation of their
writing, saw no way to explain the appearance of the new feature.  So they
denied that the senses are adequate to the understanding of reality and
insisted on the unreality of change; related to the denial of change was also
the denial of the multiplicity of things.  The Eleatic view is arguably more
subtle than what appears in this characterization, but even in classical Greece
the Eleatic view was taken by some to reject the reality of change.\footnote{%
   \cite{p2017} makes an interesting argument from Plato's {\it Parmenides}
   both for the subtlety of Parmenides' view and for the common misconception
   about what his view really was.  In any event, Aristotle argues for the
   reality of change, whether against Parmenides himself or against the common
   misunderstanding of what Parmenides' view actually was.%
}

The debate over the reality of change continues today.  Like the apparent view
of the Eleatics, a common view among physicists is that although time is real,
the \emph{passage} of time is an illusion.\footnote{%
   \cite{c2015}, for example, thinks that the ``flow'' of time from past to
   future is an illusion. \cite{s2013}, however, regards the passage of time as
   real.%
}
The universe is imagined as a four-dimensional space, one of whose dimensions
is called ``time.''  The time dimension is handled differently from the spatial
dimensions in general relativity, and so time in that sense is regarded as
perfectly real (and, at least in each particular reference frame, distinct from
space).  However, the \emph{passage} of time is like what the Eleatics called
``change'' and is now commonly regarded as unreal.

\subsection{Heraclitus}

In Ephesus---and before the time of Socrates---Heraclitus appears to have held
that everything is in flux, and nothing is really permanent.  As for the
Eleatics, there is some debate about what Heraclitus actually thought.  Like
the Eleatics, Heraclitus advocated the radical unity of all being, but his
underlying principle seems to have involved the reality of change.  In any
event, Plato and Aristotle took him to affirm the reality of change and to deny
the reality of any permanence.  The view attributed to Heraclitus is, in a
certain sense, the opposite of the apparent Eleatic view, in which there is
only permanence and no change.\footnote{%
   \cite{g2015} gives an overview of scholarship on Heraclitus.%
}

\subsection{The Two Kinds of Reality}

Aristotle proposed that in a thing that is now real in one way, there are other
things that are now real in a different way.  For example, this sphere of silly
putty is a reality that exists in one way, as an actuality.  In this actual
sphere of silly putty, there are also realities, each existing in a different
way, as a potentiality.  For example, each of
\begin{compactitem}
   \item a flat piece of silly putty and
   \item a cubic piece of silly putty
\end{compactitem}
really exists now as a potentiality in the spherical piece of silly putty.  A
real potentiality, despite its being a potentiality, is real; it really exists,
though its mode of existence is as a potency, not as an act.  Reality consists
not only of what is actual but also of what is potential in an actual thing.

In fact, the potentialities that exist in an actual thing depend on what the
actual thing is.  For example, in this sphere of silly putty there is not the
real potentiality of a zebra.  In this case, a zebra is an \emph{un}real
potentiality, one that is merely imaginable but not really in the ball of silly
putty.  Not every imaginable potentiality is really in a given actual thing.
One task of modern science is to find out what the real potentialities in every
natural substance are.

\subsection{Change}

Change occurs whenever a potentiality in a thing becomes actual.  However, a
potentiality can be actualized only by something else that is already actual.
For example, in order for the sphere of silly putty to be squashed flat, my
hands, which are actually real, must squeeze the silly putty flat.

\subsection{The Priority of the Actual}

There is an asymmetry between the actual and the potential.  One can see this
in each of a few different ways.
\begin{enumerate}
   \item A potentiality is always \emph{for} a certain kind of actuality.  To
      say that a sphere of silly putty is potentially flat just means that, if
      the right external force were applied, then the piece of silly putty
      would become actually flat.
   \item A thing's potentialities are grounded in what the thing actually is.
      If a sphere actually be made of silly putty, then that limits what it can
      potentially be.  For example, silly putty of any shape will tend, over
      the course of minutes, to flow under the force of gravity, rather than to
      hold its shape.
   \item A potentiality can be actualized only by what is actual, as we saw
      above in regard to change.
   \item A potentiality cannot exist on its own, apart from something actual.
      The potentiality to become a flat piece of silly putty resides only in a
      piece of silly putty that is actually not flat.  A potentiality must
      exist only in a thing that is otherwise actual.  That is, potentiality is
      found only in combination with actuality.
\end{enumerate}
While what is purely potential cannot (outside the mind) exist apart from what
is actual, what is purely actual can exist apart from what is potential.
Aquinas shows that what is purely actual is what one ought to call ``God.''
Further, everything else (what God creates) is a combination of actuality and
potentiality.

\subsection{Causal Powers and Laws of Nature}

Whenever one thing, such as my hand, can actualize a potentiality in something
else, such as a blob of silly putty, the first thing has a causal power.  So my
hand has the power to flatten a blob of silly putty.  In fact, every real thing
in the world, such as a stone, has causal powers to act on other things and
also inherent potentialities that can be actualized by other things.  For
example, a stone has the power to generate ripples in the surface of a pond,
and in a stone is the real potentiality of a smaller stone, which would result
from the initial stone's being broken into two pieces.  The fact that
everything has some inherent potentialities as well as actual powers gives rise
to patterns of activity in the world.

A so-called ``law of nature'' is not what makes things behave as they do.
Rather, each thing by nature has certain potentialities that can be actualized
and certain powers to actualize potencies in other things.  A law of nature is
what we propose in order to summarize the interactions of things according to
their natures.\footnote{%
   \citet[Section 1.2.2.4]{f2014} provides a good overview of causal powers and
   laws of nature.
}

\subsection{Summary}

In a real being, the distinction between what really exists as an actuality and
what really exists as a potentiality allows one to see how both permanence and
change are real:  Any actuality that persists has some permanence, and the
actualization of any potentiality is a change.

In what follows, we shall first consider the extrinsic causes of being, and
then we shall consider the intrinsic causes of being.  Each of these is further
divided into a cause representing actuality and a cause representing
potentiality. So there are four ultimate causes of being.  We begin the
extrinsic causes because one of them is most similar to the modern notion of
a cause.

\section{The Causes of Becoming}

In any change, a potentiality is actualized by something else already actual.
That thing is the \emph{efficient cause}.  For example, the combination of my
two hands and their squeezing together is the efficient cause of the silly
putty's being squashed into a flat piece.

Regardless of what is between my hands, if it be sufficiently pliable, the same
combination of my two hands and their squeezing together will tend to produce
flatness in whatever lies between them.  So an efficient cause points toward a
particular effect.  In the example, bringing about flatness is the \emph{final
cause}, that toward which the efficient cause is directed.  Note that although
I probably do intend to flatten the silly putty when I squeeze it, I need have
no such intention.  If I squeeze it by habit while thinking about something
else, then the efficient cause will still point to the same result.  That is,
\emph{bringing about the final cause} is intrinsic in the efficient cause.  The
identification of a final cause in any change is \emph{teleology}, or the study
of directedness toward an end.

\subsection{Act and Potency}

Efficient cause is to final cause as the actual is to the potential.

\subsection{Secondary Causality}

According to a common misconception, teleology can be a right analysis only if
in the efficient cause there be an intelligence that intends the end.  In this
misconception, there are only two ways for a final cause to be active in the
example of the flat blob of silly putty:
\begin{enumerate}
   \item for me to desire flatness in the silly putty and for my whole person
      to be identified as the only real efficient cause or
   \item for God to desire flatness in the silly putty and for God to be
      identified as the only real efficient cause.
\end{enumerate}
In the modern misconception, if neither of these be the case, then there is no
final cause, no valid teleology.  However, the modern misconception is wrong
for at least a couple of reasons.

First, it is too restrictive to say that my whole person must be viewed as the
efficient cause of the silly putty's becoming flat.  While it is reasonable to
consider myself as the efficient cause, especially if I intend to flatten the
silly putty, it is also reasonable to consider just the minimal physical
system, consisting of my hands and the silly putty.  In each way of looking at
what's going on, there are efficient and final causes acting.  A sculptor may
be considered in one sense the efficient cause of a sculpture.  In a more
precise sense, the chisel, together with the force applied to it, is the
efficient cause of the removal of stone from the block.  Even in that case, the
application of force to a chisel always naturally tends to remove a piece of
whatever the chisel touches.  Chiseling points toward removal of stuff.  So
there is final causality even on the lowest scale that we wish to consider.
And this extends to things wholly unconnected with human beings.

\subsection{Why To Oppose Intelligent Design}

\section{The Causes of Being}

\subsection{Act and Potency}

Formal cause is to material cause as the actual is to the potential.

\section{Creation}

\subsection{Act and Potency}

Existence is to essence as the actual is to the potential.

\bibliographystyle{plainnat}

\begin{thebibliography}{}

   \begin{small}

   \bibitem[Ali and Spencer(2003)Ali and Spencer]{as2003}
      Ali, D. and R.~Spencer\ \ {\it Inside Islam: A Guide for Catholics, 100
      Questions and Answers.}  Ascension Press.  West Chester, PA.  2003.

   \bibitem[Belloc(1938)Belloc]{b1938}
      Belloc, H.\ \ {\it The Great Heresies.}\footnote{%
         \url{http://onlinebooks.library.upenn.edu/webbin/metabook?id=heresies}
      }
      Sheed and Ward.  London.  1938.

   \bibitem[Burtt(1954)Burtt]{b1954}
      Burtt, E. A.\ \ {\it The Metaphysical Foundations of Modern Physical
      Science.}  Second Revised Edition.  Doubleday.  Garden City, NY.  1954.

   \bibitem[Carroll(2015)Carroll]{c2015}
      Carroll, S.\ \ ``The Reality of Time.''\footnote{%
         \url{http://www.preposterousuniverse.com/blog/2015/04/03/the-reality-of-time}%
      }
      On Carroll's blog.\footnote{\url{http://preposterousuniverse.com}}  2015.

   \bibitem[Chakravartty(2016)Chakravartty]{c2016}
      Chakravartty, A.\ \ ``Scientific Realism.''\footnote{%
         \url{https://plato.stanford.edu/archives/win2016/entries/scientific-realism}%
      }
      In {\it The Stanford Encyclopedia of Philosophy.}  Winter 2016 Edition.
      Metaphysics Research Lab.  Stanford University.  2016.

   \bibitem[Dawkins(2006)Dawkins]{d2006}
      Dawkins, R.\ \ ``The God Delusion.''  Houghton Mifflin Company.  New
      York.  2008.

   \bibitem[English Dominican Fathers(1920)English Dominican Fathers]{e1920}
      Fathers of the English Dominican Province, Translated by. {\it The
      Summa Theologiae of St.~Thomas Aquinas.}\footnote{%
         \url{http://www.newadvent.org/summa}%
      }
      Second and Revised Edition.  1920.

   \bibitem[Feser(2008)Feser]{f2008}
      Feser, E.\ \ {\it The Last Superstition: A Refutation of the New
      Atheism.}  St.~Augustine's Press.  2008.

   \bibitem[Feser(2012)Feser]{f2012}
      Feser, E.\ \ ``Oerter Contra the Principle of Causality.''\footnote{%
         \url{http://edwardfeser.blogspot.com/2012/05/oerter-contra-principle-of-causality.html}
      }
      On Feser's blog.\footnote{\url{http://edwardfeser.blogspot.com}}  2012.

   \bibitem[Feser(2014)Feser]{f2014}
      Feser, E.\ \ {\it Scholastic Metaphysics: A Contemporary Introduction.}
      editiones scholasticae.  Heusenstamm.  2014.

   \bibitem[Gottlieb(2015)Gottlieb]{go2015}
      Gottlieb, P.\ \ ``Aristotle on Non-contradiction.''\footnote{%
         \url{https://plato.stanford.edu/archives/sum2015/entries/aristotle-noncontradiction}%
      }
      In {\it The Stanford Encyclopedia of Philosophy.}  Summer 2015 Edition.
      Metaphysics Research Lab.  Stanford University.  2015.

   \bibitem[Graham(2015)Graham]{g2015}
      Graham, D. W.\ \ ``Heraclitus.''\footnote{%
         \url{https://plato.stanford.edu/archives/fall2015/entries/heraclitus}
      }
      In {\it The Stanford Encyclopedia of Philosophy.}  Fall 2015 Edition.
      Metaphysics Research Lab.  Stanford University.  2015.

   \bibitem[Hutchinson(2011)Hutchinson]{h2011}
      Hutchinson, I.\ \ {\it Monopolizing Knowledge: A Scientist Refutes
      Religion-Denying, Reason-Destroying Scientism.}\footnote{%
         \url{http://monopolizingknowledge.net/contents.html}.%
      }
      Fias Publishing.  Belmont, MA.  2011.

   \bibitem[Palmer(2017)Palmer]{p2017}
      Palmer, J.\ \ ``Zeno of Elea.''\footnote{%
         \url{https://plato.stanford.edu/archives/spr2017/entries/zeno-elea}%
      }
      In {\it The Stanford Encyclopedia of Philosophy.}  Spring 2017 Edition.
      Metaphysics Research Lab.  Stanford University.  2017.

   \bibitem[Ratzinger(2005)Ratzinger]{r2005}
      Ratzinger, J.\ \ {\it Truth and Tolerance: Christian Belief and World
      Religions.}  Ignatius Press.  San Francisco.  2004.

   \bibitem[Ratzinger(2007)Ratzinger]{r2007}
      Ratzinger, J.\ \ {\it Jesus of Nazareth.}  Doubleday.  New York.  2007.

   \bibitem[Smolin(2013)Smolin]{s2013}
      Smolin, L.\ \ {\it Time Reborn.}  Houghton Mifflin Harcourt Publishing
      Company.  New York.  2013.

   \bibitem[Sorell(1991)Sorell]{s1991}
      Sorell, T.\ \ {\it Scientism: Philosophy and the Infatuation with
      Science.}  Routledge.  New York.  1991.

   \bibitem[van Inwagen and Sullivan(2017)van Inwagen and Sullivan]{vis2017}
      van Inwagen, P. and M. Sullivan\ \ ``Metaphysics''\footnote{%
         \url{https://plato.stanford.edu/archives/spr2017/entries/metaphysics}%
      }
      In {\it The Stanford Encyclopedia of Philosophy.}  Spring 2017 Edition.
      Metaphysics Research Lab.  Stanford University.  2017.

   \bibitem[Vander Heeren(1912)Vander Heeren]{vdh1912}
      Vander Heeren, A.\ \ ``Septuagint Version.''\footnote{%
         \url{http://www.newadvent.org/cathen/13722a.htm}%
      }
      In {\it The Catholic Encyclopedia.}  1913 Edition.  Robert Appleton
      Company.  New York.  1912.

   \end{small}

\end{thebibliography}


\begin{footnotesize}

\section*{\rlap{GNU Free Documentation License}}
\phantomsection% so hyperref creates bookmarks
\addcontentsline{toc}{chapter}{GNU Free Documentation License}
%\label{label_fdl}

 \begin{center}

 Version 1.3, 3 November 2008

 Copyright \copyright{} 2000, 2001, 2002, 2007, 2008  Free Software
 Foundation, Inc.
 
 \bigskip
 
 \url{http://fsf.org/}
  
 \bigskip
 
 Everyone is permitted to copy and distribute verbatim copies
 of this license document, but changing it is not allowed.
\end{center}


\begin{center}
{\bf\large Preamble}
\end{center}

The purpose of this License is to make a manual, textbook, or other
functional and useful document ``free'' in the sense of freedom: to
assure everyone the effective freedom to copy and redistribute it,
with or without modifying it, either commercially or noncommercially.
Secondarily, this License preserves for the author and publisher a way
to get credit for their work, while not being considered responsible
for modifications made by others.

This License is a kind of ``copyleft'', which means that derivative
works of the document must themselves be free in the same sense.  It
complements the GNU General Public License, which is a copyleft
license designed for free software.

We have designed this License in order to use it for manuals for free
software, because free software needs free documentation: a free
program should come with manuals providing the same freedoms that the
software does.  But this License is not limited to software manuals;
it can be used for any textual work, regardless of subject matter or
whether it is published as a printed book.  We recommend this License
principally for works whose purpose is instruction or reference.


\begin{center}
{\Large\bf 1. APPLICABILITY AND DEFINITIONS\par}
\phantomsection%
\addcontentsline{toc}{section}{1. APPLICABILITY AND DEFINITIONS}
\end{center}

This License applies to any manual or other work, in any medium, that
contains a notice placed by the copyright holder saying it can be
distributed under the terms of this License.  Such a notice grants a
world-wide, royalty-free license, unlimited in duration, to use that
work under the conditions stated herein.  The ``\textbf{Document}'', below,
refers to any such manual or work.  Any member of the public is a
licensee, and is addressed as ``\textbf{you}''.  You accept the license if you
copy, modify or distribute the work in a way requiring permission
under copyright law.

A ``\textbf{Modified Version}'' of the Document means any work containing the
Document or a portion of it, either copied verbatim, or with
modifications and/or translated into another language.

A ``\textbf{Secondary Section}'' is a named appendix or a front-matter section of
the Document that deals exclusively with the relationship of the
publishers or authors of the Document to the Document's overall subject
(or to related matters) and contains nothing that could fall directly
within that overall subject.  (Thus, if the Document is in part a
textbook of mathematics, a Secondary Section may not explain any
mathematics.)  The relationship could be a matter of historical
connection with the subject or with related matters, or of legal,
commercial, philosophical, ethical or political position regarding
them.

The ``\textbf{Invariant Sections}'' are certain Secondary Sections whose titles
are designated, as being those of Invariant Sections, in the notice
that says that the Document is released under this License.  If a
section does not fit the above definition of Secondary then it is not
allowed to be designated as Invariant.  The Document may contain zero
Invariant Sections.  If the Document does not identify any Invariant
Sections then there are none.

The ``\textbf{Cover Texts}'' are certain short passages of text that are listed,
as Front-Cover Texts or Back-Cover Texts, in the notice that says that
the Document is released under this License.  A Front-Cover Text may
be at most 5 words, and a Back-Cover Text may be at most 25 words.

A ``\textbf{Transparent}'' copy of the Document means a machine-readable copy,
represented in a format whose specification is available to the
general public, that is suitable for revising the document
straightforwardly with generic text editors or (for images composed of
pixels) generic paint programs or (for drawings) some widely available
drawing editor, and that is suitable for input to text formatters or
for automatic translation to a variety of formats suitable for input
to text formatters.  A copy made in an otherwise Transparent file
format whose markup, or absence of markup, has been arranged to thwart
or discourage subsequent modification by readers is not Transparent.
An image format is not Transparent if used for any substantial amount
of text.  A copy that is not ``Transparent'' is called ``\textbf{Opaque}''.

Examples of suitable formats for Transparent copies include plain ASCII without
markup, Texinfo input format, LaTeX input format, SGML or XML using a publicly
available DTD, and standard-conforming simple HTML, PostScript or PDF designed
for human modification.  Examples of transparent image formats include PNG, XCF
and JPG\@.  Opaque formats include proprietary formats that can be read and
edited only by proprietary word processors, SGML or XML for which the DTD
and/or processing tools are not generally available, and the machine-generated
HTML, PostScript or PDF produced by some word processors for output purposes
only.

The ``\textbf{Title Page}'' means, for a printed book, the title page itself,
plus such following pages as are needed to hold, legibly, the material
this License requires to appear in the title page.  For works in
formats which do not have any title page as such, ``Title Page'' means
the text near the most prominent appearance of the work's title,
preceding the beginning of the body of the text.

The ``\textbf{publisher}'' means any person or entity that distributes
copies of the Document to the public.

A section ``\textbf{Entitled XYZ}'' means a named subunit of the Document whose
title either is precisely XYZ or contains XYZ in parentheses following
text that translates XYZ in another language.  (Here XYZ stands for a
specific section name mentioned below, such as ``\textbf{Acknowledgements}'',
``\textbf{Dedications}'', ``\textbf{Endorsements}'', or ``\textbf{History}''.)  
To ``\textbf{Preserve the Title}''
of such a section when you modify the Document means that it remains a
section ``Entitled XYZ'' according to this definition.

The Document may include Warranty Disclaimers next to the notice which
states that this License applies to the Document.  These Warranty
Disclaimers are considered to be included by reference in this
License, but only as regards disclaiming warranties: any other
implication that these Warranty Disclaimers may have is void and has
no effect on the meaning of this License.


\begin{center}
{\Large\bf 2. VERBATIM COPYING\par}
\phantomsection%
\addcontentsline{toc}{section}{2. VERBATIM COPYING}
\end{center}

You may copy and distribute the Document in any medium, either
commercially or noncommercially, provided that this License, the
copyright notices, and the license notice saying this License applies
to the Document are reproduced in all copies, and that you add no other
conditions whatsoever to those of this License.  You may not use
technical measures to obstruct or control the reading or further
copying of the copies you make or distribute.  However, you may accept
compensation in exchange for copies.  If you distribute a large enough
number of copies you must also follow the conditions in section~3.

You may also lend copies, under the same conditions stated above, and
you may publicly display copies.


\begin{center}
{\Large\bf 3. COPYING IN QUANTITY\par}
\phantomsection%
\addcontentsline{toc}{section}{3. COPYING IN QUANTITY}
\end{center}


If you publish printed copies (or copies in media that commonly have
printed covers) of the Document, numbering more than 100, and the
Document's license notice requires Cover Texts, you must enclose the
copies in covers that carry, clearly and legibly, all these Cover
Texts: Front-Cover Texts on the front cover, and Back-Cover Texts on
the back cover.  Both covers must also clearly and legibly identify
you as the publisher of these copies.  The front cover must present
the full title with all words of the title equally prominent and
visible.  You may add other material on the covers in addition.
Copying with changes limited to the covers, as long as they preserve
the title of the Document and satisfy these conditions, can be treated
as verbatim copying in other respects.

If the required texts for either cover are too voluminous to fit
legibly, you should put the first ones listed (as many as fit
reasonably) on the actual cover, and continue the rest onto adjacent
pages.

If you publish or distribute Opaque copies of the Document numbering
more than 100, you must either include a machine-readable Transparent
copy along with each Opaque copy, or state in or with each Opaque copy
a computer-network location from which the general network-using
public has access to download using public-standard network protocols
a complete Transparent copy of the Document, free of added material.
If you use the latter option, you must take reasonably prudent steps,
when you begin distribution of Opaque copies in quantity, to ensure
that this Transparent copy will remain thus accessible at the stated
location until at least one year after the last time you distribute an
Opaque copy (directly or through your agents or retailers) of that
edition to the public.

It is requested, but not required, that you contact the authors of the
Document well before redistributing any large number of copies, to give
them a chance to provide you with an updated version of the Document.


\begin{center}
{\Large\bf 4. MODIFICATIONS\par}
\phantomsection%
\addcontentsline{toc}{section}{4. MODIFICATIONS}
\end{center}

You may copy and distribute a Modified Version of the Document under
the conditions of sections 2 and 3 above, provided that you release
the Modified Version under precisely this License, with the Modified
Version filling the role of the Document, thus licensing distribution
and modification of the Modified Version to whoever possesses a copy
of it.  In addition, you must do these things in the Modified Version:

\begin{itemize}
\item[A.] 
   Use in the Title Page (and on the covers, if any) a title distinct
   from that of the Document, and from those of previous versions
   (which should, if there were any, be listed in the History section
   of the Document).  You may use the same title as a previous version
   if the original publisher of that version gives permission.
   
\item[B.]
   List on the Title Page, as authors, one or more persons or entities
   responsible for authorship of the modifications in the Modified
   Version, together with at least five of the principal authors of the
   Document (all of its principal authors, if it has fewer than five),
   unless they release you from this requirement.
   
\item[C.]
   State on the Title page the name of the publisher of the
   Modified Version, as the publisher.
   
\item[D.]
   Preserve all the copyright notices of the Document.
   
\item[E.]
   Add an appropriate copyright notice for your modifications
   adjacent to the other copyright notices.
   
\item[F.]
   Include, immediately after the copyright notices, a license notice
   giving the public permission to use the Modified Version under the
   terms of this License, in the form shown in the Addendum below.
   
\item[G.]
   Preserve in that license notice the full lists of Invariant Sections
   and required Cover Texts given in the Document's license notice.
   
\item[H.]
   Include an unaltered copy of this License.
   
\item[I.]
   Preserve the section Entitled ``History'', Preserve its Title, and add
   to it an item stating at least the title, year, new authors, and
   publisher of the Modified Version as given on the Title Page.  If
   there is no section Entitled ``History'' in the Document, create one
   stating the title, year, authors, and publisher of the Document as
   given on its Title Page, then add an item describing the Modified
   Version as stated in the previous sentence.
   
\item[J.]
   Preserve the network location, if any, given in the Document for
   public access to a Transparent copy of the Document, and likewise
   the network locations given in the Document for previous versions
   it was based on.  These may be placed in the ``History'' section.
   You may omit a network location for a work that was published at
   least four years before the Document itself, or if the original
   publisher of the version it refers to gives permission.
   
\item[K.]
   For any section Entitled ``Acknowledgements'' or ``Dedications'',
   Preserve the Title of the section, and preserve in the section all
   the substance and tone of each of the contributor acknowledgements
   and/or dedications given therein.
   
\item[L.]
   Preserve all the Invariant Sections of the Document,
   unaltered in their text and in their titles.  Section numbers
   or the equivalent are not considered part of the section titles.
   
\item[M.]
   Delete any section Entitled ``Endorsements''.  Such a section
   may not be included in the Modified Version.
   
\item[N.]
   Do not retitle any existing section to be Entitled ``Endorsements''
   or to conflict in title with any Invariant Section.
   
\item[O.]
   Preserve any Warranty Disclaimers.
\end{itemize}

If the Modified Version includes new front-matter sections or
appendices that qualify as Secondary Sections and contain no material
copied from the Document, you may at your option designate some or all
of these sections as invariant.  To do this, add their titles to the
list of Invariant Sections in the Modified Version's license notice.
These titles must be distinct from any other section titles.

You may add a section Entitled ``Endorsements'', provided it contains
nothing but endorsements of your Modified Version by various
parties---for example, statements of peer review or that the text has
been approved by an organization as the authoritative definition of a
standard.

You may add a passage of up to five words as a Front-Cover Text, and a
passage of up to 25 words as a Back-Cover Text, to the end of the list
of Cover Texts in the Modified Version.  Only one passage of
Front-Cover Text and one of Back-Cover Text may be added by (or
through arrangements made by) any one entity.  If the Document already
includes a cover text for the same cover, previously added by you or
by arrangement made by the same entity you are acting on behalf of,
you may not add another; but you may replace the old one, on explicit
permission from the previous publisher that added the old one.

The author (authors) and publisher (publishers) of the Document do not by this
License give permission to use their names for publicity for or to assert or
imply endorsement of any Modified Version.


\begin{center}
{\Large\bf 5. COMBINING DOCUMENTS\par}
\phantomsection%
\addcontentsline{toc}{section}{5. COMBINING DOCUMENTS}
\end{center}


You may combine the Document with other documents released under this
License, under the terms defined in section~4 above for modified
versions, provided that you include in the combination all of the
Invariant Sections of all of the original documents, unmodified, and
list them all as Invariant Sections of your combined work in its
license notice, and that you preserve all their Warranty Disclaimers.

The combined work need only contain one copy of this License, and
multiple identical Invariant Sections may be replaced with a single
copy.  If there are multiple Invariant Sections with the same name but
different contents, make the title of each such section unique by
adding at the end of it, in parentheses, the name of the original
author or publisher of that section if known, or else a unique number.
Make the same adjustment to the section titles in the list of
Invariant Sections in the license notice of the combined work.

In the combination, you must combine any sections Entitled ``History''
in the various original documents, forming one section Entitled
``History''; likewise combine any sections Entitled ``Acknowledgements'',
and any sections Entitled ``Dedications''.  You must delete all sections
Entitled ``Endorsements''.

\begin{center}
{\Large\bf 6. COLLECTIONS OF DOCUMENTS\par}
\phantomsection%
\addcontentsline{toc}{section}{6. COLLECTIONS OF DOCUMENTS}
\end{center}

You may make a collection consisting of the Document and other documents
released under this License, and replace the individual copies of this
License in the various documents with a single copy that is included in
the collection, provided that you follow the rules of this License for
verbatim copying of each of the documents in all other respects.

You may extract a single document from such a collection, and distribute
it individually under this License, provided you insert a copy of this
License into the extracted document, and follow this License in all
other respects regarding verbatim copying of that document.


\begin{center}
{\Large\bf 7. AGGREGATION WITH INDEPENDENT WORKS\par}
\phantomsection%
\addcontentsline{toc}{section}{7. AGGREGATION WITH INDEPENDENT WORKS}
\end{center}


A compilation of the Document or its derivatives with other separate
and independent documents or works, in or on a volume of a storage or
distribution medium, is called an ``aggregate'' if the copyright
resulting from the compilation is not used to limit the legal rights
of the compilation's users beyond what the individual works permit.
When the Document is included in an aggregate, this License does not
apply to the other works in the aggregate which are not themselves
derivative works of the Document.

If the Cover Text requirement of section~3 is applicable to these
copies of the Document, then if the Document is less than one half of
the entire aggregate, the Document's Cover Texts may be placed on
covers that bracket the Document within the aggregate, or the
electronic equivalent of covers if the Document is in electronic form.
Otherwise they must appear on printed covers that bracket the whole
aggregate.


\begin{center}
{\Large\bf 8. TRANSLATION\par}
\phantomsection%
\addcontentsline{toc}{section}{8. TRANSLATION}
\end{center}


Translation is considered a kind of modification, so you may
distribute translations of the Document under the terms of section~4.
Replacing Invariant Sections with translations requires special
permission from their copyright holders, but you may include
translations of some or all Invariant Sections in addition to the
original versions of these Invariant Sections.  You may include a
translation of this License, and all the license notices in the
Document, and any Warranty Disclaimers, provided that you also include
the original English version of this License and the original versions
of those notices and disclaimers.  In case of a disagreement between
the translation and the original version of this License or a notice
or disclaimer, the original version will prevail.

If a section in the Document is Entitled ``Acknowledgements'',
``Dedications'', or ``History'', the requirement (section~4) to Preserve
its Title (section~1) will typically require changing the actual
title.


\begin{center}
{\Large\bf 9. TERMINATION\par}
\phantomsection%
\addcontentsline{toc}{section}{9. TERMINATION}
\end{center}


You may not copy, modify, sublicense, or distribute the Document
except as expressly provided under this License.  Any attempt
otherwise to copy, modify, sublicense, or distribute it is void, and
will automatically terminate your rights under this License.

However, if you cease all violation of this License, then your license
from a particular copyright holder is reinstated (a) provisionally,
unless and until the copyright holder explicitly and finally
terminates your license, and (b) permanently, if the copyright holder
fails to notify you of the violation by some reasonable means prior to
60 days after the cessation.

Moreover, your license from a particular copyright holder is
reinstated permanently if the copyright holder notifies you of the
violation by some reasonable means, this is the first time you have
received notice of violation of this License (for any work) from that
copyright holder, and you cure the violation prior to 30 days after
your receipt of the notice.

Termination of your rights under this section does not terminate the
licenses of parties who have received copies or rights from you under
this License.  If your rights have been terminated and not permanently
reinstated, receipt of a copy of some or all of the same material does
not give you any rights to use it.


\begin{center}
{\Large\bf 10. FUTURE REVISIONS OF THIS LICENSE\par}
\phantomsection%
\addcontentsline{toc}{section}{10. FUTURE REVISIONS OF THIS LICENSE}
\end{center}


The Free Software Foundation may publish new, revised versions
of the GNU Free Documentation License from time to time.  Such new
versions will be similar in spirit to the present version, but may
differ in detail to address new problems or concerns.  See
http://www.gnu.org/copyleft/.

Each version of the License is given a distinguishing version number.
If the Document specifies that a particular numbered version of this
License ``or any later version'' applies to it, you have the option of
following the terms and conditions either of that specified version or
of any later version that has been published (not as a draft) by the
Free Software Foundation.  If the Document does not specify a version
number of this License, you may choose any version ever published (not
as a draft) by the Free Software Foundation.  If the Document
specifies that a proxy can decide which future versions of this
License can be used, that proxy's public statement of acceptance of a
version permanently authorizes you to choose that version for the
Document.


\begin{center}
{\Large\bf 11. RELICENSING\par}
\phantomsection%
\addcontentsline{toc}{section}{11. RELICENSING}
\end{center}


``Massive Multiauthor Collaboration Site'' (or ``MMC Site'') means any
World Wide Web server that publishes copyrightable works and also
provides prominent facilities for anybody to edit those works.  A
public wiki that anybody can edit is an example of such a server.  A
``Massive Multiauthor Collaboration'' (or ``MMC'') contained in the
site means any set of copyrightable works thus published on the MMC
site.

``CC-BY-SA'' means the Creative Commons Attribution-Share Alike 3.0
license published by Creative Commons Corporation, a not-for-profit
corporation with a principal place of business in San Francisco,
California, as well as future copyleft versions of that license
published by that same organization.

``Incorporate'' means to publish or republish a Document, in whole or
in part, as part of another Document.

An MMC is ``eligible for relicensing'' if it is licensed under this
License, and if all works that were first published under this License
somewhere other than this MMC, and subsequently incorporated in whole
or in part into the MMC, (1) had no cover texts or invariant sections,
and (2) were thus incorporated prior to November 1, 2008.

The operator of an MMC Site may republish an MMC contained in the site
under CC-BY-SA on the same site at any time before August 1, 2009,
provided the MMC is eligible for relicensing.


\begin{center}
{\Large\bf ADDENDUM\@: How to use this License for your documents\par}
\phantomsection%
\addcontentsline{toc}{section}{ADDENDUM\@: How to use this License for your
documents}
\end{center}

To use this License in a document you have written, include a copy of
the License in the document and put the following copyright and
license notices just after the title page:

\bigskip
\begin{quote}
    Copyright \copyright{}  YEAR  YOUR NAME\@.
    Permission is granted to copy, distribute and/or modify this document
    under the terms of the GNU Free Documentation License, Version 1.3
    or any later version published by the Free Software Foundation;
    with no Invariant Sections, no Front-Cover Texts, and no Back-Cover Texts.
    A copy of the license is included in the section entitled ``GNU
    Free Documentation License''.
\end{quote}
\bigskip
    
If you have Invariant Sections, Front-Cover Texts and Back-Cover Texts,
replace the ``with \dots\ Texts.'' line with this:

\bigskip
\begin{quote}
    with the Invariant Sections being LIST THEIR TITLES, with the Front-Cover
    Texts being LIST, and with the Back-Cover Texts being LIST\@.
\end{quote}
\bigskip
    
If you have Invariant Sections without Cover Texts, or some other
combination of the three, merge those two alternatives to suit the
situation.

If your document contains nontrivial examples of program code, we
recommend releasing these examples in parallel under your choice of
free software license, such as the GNU General Public License,
to permit their use in free software.

\end{footnotesize}



\end{document}

