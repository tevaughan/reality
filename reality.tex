
\newcommand{\doctitle}{How Things Really Are, Actually and Potentially}

\documentclass[twocolumn]{article}

\usepackage{fancyhdr}
\usepackage{ftnright}
\usepackage{graphicx}
\usepackage{lastpage}
\usepackage{natbib}
\usepackage{times}
\usepackage{vmargin}

% hyperref must be last package.
\usepackage[colorlinks=true,citecolor=blue,hyperfootnotes=false]{hyperref}

% vmargin setup
\setpapersize{USletter}
\setmarginsrb%
{0.4in}%           left
{0.4in}%           top
{0.4in}%           right
{0.5in}%           bottom
{2\baselineskip}%  headheight
{2\baselineskip}%  headsep
{3\baselineskip}%  footheight
{4\baselineskip}%  footskip

% mydate macro
\newcommand{\mydate}{%
   \number\year\space%
   \ifcase\month\or%
      Jan\or\ Feb\or\ Mar\or\ Apr\or\ May\or\ Jun\or%
      Jul\or\ Aug\or\ Sep\or\ Oct\or\ Nov\or\ Dec
   \fi\space%
   \number\day%
}

% fancyhdr settings
\pagestyle{fancy}
\lhead{\sffamily\textbf{\doctitle}}
\chead{}
\rhead{\sffamily \thepage~of~\pageref{LastPage}}
\renewcommand{\headrulewidth}{1pt}
\renewcommand{\footrulewidth}{1pt}
\lfoot{%
   \scriptsize\sffamily
   \begin{minipage}{0.95\textwidth}
   Copyright~\copyright~2017~Thomas E. Vaughan.\ \ \
   PDF image generated on \mydate.\ \ \
   Source code and on-line PDF version at
      \url{https://github.com/tevaughan/reality}.\ \ \
   Permission is granted to copy, distribute and/or modify this document under
   the terms of the GNU Free Documentation License, Version 1.3 or any later
   version published by the Free Software Foundation; with no Invariant
   Sections, no Front-Cover Texts, and no Back-Cover Texts.  A copy of the
   license is included in the section entitled ``GNU Free Documentation
   License''.
   \end{minipage}%
}
\cfoot{}
\rfoot{%
   \begin{minipage}{0.05\textwidth}
   \begin{flushright}
   \includegraphics[width=0.85\textwidth]{logo}
   \end{flushright}
   \end{minipage}%
}

\renewcommand{\footnoterule}{%
   \kern -3pt
   \hrule width 0.5 \columnwidth
   \kern 2.6pt
}

\begin{document}

\thispagestyle{fancy}

\section{Introduction}

There is a great war raging.  Western civilization is beseiged.  By ``Western
Civilization'' I mean the civilization that grew out of the marriage between
ancient Israel and classical Greece.  Like any living thing in the world, she
has struggled to survive since the moment of her birth.  However, Western
civilization makes some broad claims about what is objectively true and good.
So, in addition to enemies who might naturally oppose any civilization, the
West has her particular enemies.  In this talk, I shall focus on the Western
view of reality---in light of the teaching of St.~Thomas of Aquinas---and on a
particular opposition to this view.

Let us reflect on some points of the West's development.  First, the Hebrew and
the Greek traditions became intertwined in the writing of the {\it
Septuagint}.\footnote{%
   The {\it Septuagint} is the first translation of the {\it Tanakh} (what
   Christians call the {\it Old Testament}).  The translation, into Greek, was
   made before the Christian era.  \cite{vdh1912} provides a detailed overview.
   \cite{r2005, r2007} comments on the marriage of the Greek and the Hebrew
   cultures in the writing of the {\it Septuagint}.
}
The two became one in Christianity, which required both traditions in order to
express doctrine.  Christians converted the declining Roman empire to the
Faith, defended Europe against the onslaught of Muslim invaders,\footnote{%
   \citet[Chapter 4]{b1938} gives a concise overview of the circumstances
   leading up to the First Crusade.
}
and finally established a new civilization, among whose fruits are modern
science\footnote{%
   The terms ``science'' and ``scientific,'' by etymology, could be misleading.
   In Latin, the verb ``scire'' means ``to know,'' but many a scientific theory
   cannot be known as a truth.  Modern science does have truths: Every
   repeatable perception resulting from a carefully described experiment or
   observation is a truth that has the power to prove a theory false, even a
   theory that had long been treated as standard.  However, a scientific theory
   that refers to something imperceptible, like the electron, might some day be
   superseded by a new theory making no reference to the imperceptible thing.
   The imperceptible thing was proposed as a possible reason for the
   perceptible result of an experiment.  The imperceptible thing's existence is
   not certain, and so the theory referring to it cannot be known as a truth.
   There are many who would claim that the electron's existence is certain, but
   such ``entity realism'' has problems.  \cite{c2016} gives an overview of
   scientific realism.%
}
and laws respecting the rights of man.  Western civilization is not identical
with Christian civilization; there are, for example, Jews who participate fully
in the Western tradition.\footnote{%
   A person who calls himself ``Muslim'' might participate fully in Western
   culture.  However, if his support for universal human rights be grounded in
   the belief that man is created in the image of God, then he participates in
   Western culture against his own religion, which condemns the Judeo-Christian
   idea of man.  Even if he find some other basis for universal human rights,
   he will have trouble reconciling it with Islamic tradition.  \cite{as2003}
   point out that both the Sunni and the Shiite traditions in Islam are opposed
   to universal human rights (unless the only humans in existence were
   Muslims).
}
Still, her existence is principally due to Christianity.  An enemy opposing her
goodness might deny universal human dignity.  An enemy opposing her truth might
deny the principle of non-contradiction.\footnote{%
   \cite{go2015} summarizes Aristotle's view on the principle of
   non-contradiction.
}
The West has many enemies, both within countries formed by her ideals and
without.  Since the time of Descartes, who denied Aristotelian metaphysics, the
West has had philosophical enemies within her own ranks.  This has continued
down to the present day.

One enemy of Western civilization is a recent form of atheism.  New
Atheists,\footnote{%
   ``New Atheism'' is a term widely used to refer to ideas popularized by
   Richard Dawkins, Sam Harris, Daniel Dennett, and Christopher Hitchens in the
   first decade of the 21st Century.  \cite{f2008} devotes a whole book to the
   refutation of the philosophical errors of the New Atheists.
}
such as Richard Dawkins, deny both the Creator and the Aristotelian
metaphysical\footnote{%
   The word, ``metaphysics,'' originates from the order of the books in the
   traditional list of Aristotle's writings.  His writing on the distinction
   between act and potency came after (meta) his writing on nature (physics).
   These days, any theory about what lies at the root of being or of change is
   called ``a metaphysical theory.''%
}
basis for understanding reality while claiming for themselves the mantle of
science.  In the general context of the great war and the particular context of
the attack from the New Atheists, I point to the philosophical conflict
concerning how things really are.

\subsection{Ancient Israel and Creation}

The story of the conflict begins in ancient Israel.  The relevant idea from
{\it Genesis} is that things have not always existed.  God, Who is not a thing,
initially created things.  A new thing, such as a pot, might come into
existence by transformation from a pre-existing thing, such as a lump of clay.
However, according God's revelation to the Israelites, there was in the past a
beginning when the first things were created from nothing (ex nihilo), not by
transformation.  The fundamental distinction made by the ancient Israelites is
between God and what God creates.

\subsection{Classical Greece and Transformation}

Another thread of the story begins in classical Greece.  The early philosophers
struggled to understand how things change.  The difficulty of understanding
change led some to conclude that change is not real; others concluded that
there is no permanence at all. Either way, sense experience was viewed as
fundamentally illusory.  The Greek philosophical tradition culminated, however,
in the writing of Aristotle, who stood up for the reality of sense experience
by explaining that for a thing to change is for what had existed only
potentially in the thing to begin existing actually.  The fundamental
distinction is between actuality and potentiality.  This distinction is the
root of what has come to be called ``Aristotelian metaphysics,'' and it led to
modern science, which takes sense experience as the fundamental data in need of
theoretical explanation.

\subsection{Synthesis in Christianity}

When, after having been lost for a time, Aristotle was reintroduced into the
West by Saint Thomas Aquinas, he perfected the metaphysical theory so that it
could explain not only change by transformation but also creation ex nihilo.
Aquinas showed how, even if there had been no beginning in time, there would
still be creation ex nihilo at every moment in time.\footnote{%
   Aquinas shows how we cannot know by observation whether the universe have a
   finite age.  Only revelation can give us certainty that time began.  (I note
   that even a successful theory like the Big Bang cannot give us certainty
   that time began.)  However, Aquinas also shows that God creates the universe
   ex nihilo at every moment in time.  So creation ex nihilo does not require a
   beginning in time.  That is, God could create the universe such that it
   always existed, if God wanted to.  See the translation of the {\it Summa
   Theologiae} by the \citet[I, Q45, A1, and I, Q46, A1]{e1920}.
}

\subsection{Scientism, the New Atheism}

Although Aristotle's metaphysical theory naturally led to the emergence of
modern science in the West,\footnote{%
   \citet[Chapter 1]{f2014} points out that Aristotle's middle ground between
   the apparent extremes of the Eleatics and Heraclitus is precisely what is
   needed for something like modern science to work.  More generally Aristotle
   developed the idea that a physical thing has a nature that can be discovered
   through sense experience.  The incorporation of these ideas into
   Christianity through Aquinas and the Catholic Church's systematic
   investigation of miraculous claims, to see if each claim had a natural
   explanation, laid the groundwork for the emergence of modern science in the
   West.
}
Aristotle's opponents today ironically mount their attack under the banner of
science.  The self-proclaimed advocate\footnote{%
   Bill Nye, for example, is one of the most famous advocates of science.  Yet
   in advocating not merely science but scientism, he makes a philosophical
   error.  Bishop Robert Barron appears in a short movie in which he talks
   about Bill Nye's scientism:
   \url{https://www.youtube.com/watch?v=SH_Njsa0zVQ}.%
}
of science typically holds an erroneous view, \emph{scientism},\footnote{%
   Arguably the initial, definitive use of the word, ``scientism,'' is given by
   \cite{s1991}.  My definition is essentially the one given by \cite{h2011}.
   There are many arguments against scientism.  See, for example, Thomas
   Nagel's argument from the problem of qualia, or subjective experience.
   Nagel's view is summarized here:
   \url{http://www.nybooks.com/articles/2017/06/08/how-to-imagine-consciousness},
   in his response to a comment by a professor Black.  Note that Nagel is an
   atheist, though he is out of favor with the typical modern atheist, whose
   materialism takes the form of scientism.  \cite{f2014}, in the introductory
   chapter of his book, summarizes all of the main arguments against
   scientism.%
}
according to which everything that exists is describable by modern science.  We
shall see in what follows how Aristotelian metaphysics is opposed to scientism.
First, however, we must explore the key Aristotelian distinction, which allowed
Aquinas to form a synthesis of the Judeo-Christian idea of creation and the
Greek explanation of change.

\section{The Actual and the Potential}

Aristotle advanced the distinction between the actual and the potential.  He
seems to have done this initially in order to explain both the reality of
change and the reality of permanence.

\subsection{The Eleatics}

In the Greek colony of Elea in Southern Italy---and before the time of
Socrates---Parmenides and Zeno\footnote{%
   Zeno wrote many paradoxes, but what is commonly presented as ``Zeno's
   paradox'' is that idea that one cannot travel from point~A to point~B along
   a straight line.  First one would have to reach the midpoint~C between
   A~and~B, but, before that, one would have to reach the midpoint between
   A~and~C, etc.  The infinite regression was intended to show the absurdity of
   local motion.  Zeno, like Parmenides, appeared to deny both the reality of
   all change and the multiplicity of beings.%
}
appear to have held that change is unreal.  The general idea is that in any
change something new arises; what is new did not exist before the change.
Denying that creation of anything new takes place, the Eleatics, at least on a
common interpretation of their writing, saw no way to explain the appearance of
the new feature.  So they denied that the senses are adequate to the
understanding of reality and insisted on the unreality of change.  The Eleatic
view is arguably more subtle than what appears in this characterization, but
even in classical Greece the Eleatic view was taken by some to be simply the
rejection of the reality of change.\footnote{%
   \cite{p2017} makes an interesting argument from Plato's {\it Parmenides}
   both for the subtlety of Parmenides' view and for the common misconception
   about what his view really was.  In any event, Aristotle argues for the
   reality of change, whether against Parmenides himself or against the common
   misunderstanding of what Parmenides' view actually was.%
}

The debate over the reality of change continues today.  Like the apparent view
of the Eleatics, a common view among physicists is that although time is real,
the \emph{passage} of time is an illusion.\footnote{%
   \cite{c2015}, for example, thinks that the ``flow'' of time from past to
   future is an illusion. \cite{s2013}, however, regards the passage of time as
   real.%
}
The universe is imagined as a four-dimensional space, one of whose dimensions
is called ``time.''  The time dimension is handled differently from the spatial
dimensions in general relativity, and so time in that sense is regarded as
perfectly real (and, at least in each particular reference frame, distinct from
space).  However, the \emph{passage} of time is like what the Eleatics called
``change'' and is now commonly regarded as unreal.

\subsection{Heraclitus}

In Ephesus---and before the time of Socrates---Heraclitus appears to have held
that everything is in flux, and nothing is permanent.  As for the Eleatics,
there is some debate about what Heraclitus actually thought.  Nevertheless,
Plato and Aristotle took him to affirm the reality of change and to deny the
reality of any permanence.  The view attributed to Heraclitus is in a sense the
opposite of the apparent Eleatic view, in which there is only permanence and no
change.\footnote{%
   \cite{g2015} gives an overview of scholarship on Heraclitus.%
}

\subsection{The General Theory}

Aristotle proposed that there are not only real actualities in a thing but also
real potentialities in the thing.  Change occurs whenever a potentiality in a
thing becomes actual.

\subsubsection{Example: Silly Putty}

If I have a spherical ball of silly putty, then it is potentially flat.  If I
squash the ball, the potential flatness becomes actual.

\subsection{Causal Powers and Laws of Nature}

A so-called ``law of nature'' is not what makes things behave as they do.
Rather, every thing by nature has certain potentialities and some powers to
actualize potencies in other things.  A law of nature is what we propose in
order to summarize the interactions of things according to their natures.

\section{Efficient and Final Causality}

Efficient cause is to final cause as the actual is to the potential.

\section{Formal and Material Causality}

Formal cause is to material cause as the actual is to the potential.

\section{Existence and Essence}

Existence is to essence as the actual is to the potential.

\bibliographystyle{plainnat}

\begin{thebibliography}{}

   \begin{small}

   \bibitem[Ali and Spencer(2003)Ali and Spencer]{as2003}
      Ali, D. and R.~Spencer\ \ {\it Inside Islam: A Guide for Catholics, 100
      Questions and Answers.}  Ascension Press.  West Chester, PA.  2003.

   \bibitem[Belloc(1938)Belloc]{b1938}
      Belloc, H.\ \ {\it The Great Heresies.}\footnote{%
         \url{http://onlinebooks.library.upenn.edu/webbin/metabook?id=heresies}
      }
      Sheed and Ward.  London.  1938.

   \bibitem[Carroll(2015)Carroll]{c2015}
      Carroll, S.\ \ ``The Reality of Time.''\footnote{%
         \url{http://www.preposterousuniverse.com/blog/2015/04/03/the-reality-of-time}%
      }
      On Carroll's blog.\footnote{\url{http://preposterousuniverse.com}}  2015.

   \bibitem[Chakravartty(2016)Chakravartty]{c2016}
      Chakravartty, A.\ \ ``Scientific Realism.''\footnote{%
         \url{https://plato.stanford.edu/archives/win2016/entries/scientific-realism}%
      }
      In {\it The Stanford Encyclopedia of Philosophy.}  Winter 2016 Edition.
      Metaphysics Research Lab.  Stanford University.  2016.

   \bibitem[English Dominican Fathers(1920)English Dominican Fathers]{e1920}
      Fathers of the English Dominican Province, Translated by. {\it The
      Summa Theologiae of St.~Thomas Aquinas.}\footnote{%
         \url{http://www.newadvent.org/summa}%
      }
      Second and Revised Edition.  1920.

   \bibitem[Feser(2008)Feser]{f2008}
      Feser, E.\ \ {\it The Last Superstition: A Refutation of the New
      Atheism.}  St.~Augustine's Press.  2008.

   \bibitem[Feser(2014)Feser]{f2014}
      Feser, E.\ \ {\it Scholastic Metaphysics: A Contemporary Introduction.}
      editiones scholasticae.  Heusenstamm.  2014.

   \bibitem[Gottlieb(2015)Gottlieb]{go2015}
      Gottlieb, Paula, ``Aristotle on Non-contradiction.''\footnote{%
         \url{https://plato.stanford.edu/archives/sum2015/entries/aristotle-noncontradiction}%
      }
      In {\it The Stanford Encyclopedia of Philosophy.}  Summer 2015 Edition.
      Metaphysics Research Lab.  Stanford University.  2015.

   \bibitem[Graham(2015)Graham]{g2015}
      Graham, D. W.\ \ ``Heraclitus.''\footnote{%
         \url{https://plato.stanford.edu/archives/fall2015/entries/heraclitus}
      }
      In {\it The Stanford Encyclopedia of Philosophy.}  Fall 2015 Edition.
      Metaphysics Research Lab.  Stanford University.  2015.

   \bibitem[Hutchinson(2011)Hutchinson]{h2011}
      Hutchinson, I.\ \ {\it Monopolizing Knowledge: A Scientist Refutes
      Religion-Denying, Reason-Destroying Scientism.}\footnote{%
         \url{http://monopolizingknowledge.net/contents.html}.%
      }
      Fias Publishing.  Belmont, MA.  2011.

   \bibitem[Palmer(2017)Palmer]{p2017}
      Palmer, J.\ \ ``Zeno of Elea.''\footnote{%
         \url{https://plato.stanford.edu/archives/spr2017/entries/zeno-elea}%
      }
      In {\it The Stanford Encyclopedia of Philosophy.}  Spring 2017 Edition.
      Metaphysics Research Lab.  Stanford University.  2017.

   \bibitem[Ratzinger(2005)Ratzinger]{r2005}
      Ratzinger, J.\ \ {\it Truth and Tolerance: Christian Belief and World
      Religions.}  Ignatius Press.  San Francisco.  2004.

   \bibitem[Ratzinger(2007)Ratzinger]{r2007}
      Ratzinger, J.\ \ {\it Jesus of Nazareth.}  Doubleday.  New York.  2007.

   \bibitem[Smolin(2013)Smolin]{s2013}
      Smolin, L.\ \ {\it Time Reborn.}  Houghton Mifflin Harcourt Publishing
      Company.  New York.  2013.

   \bibitem[Sorell(1991)Sorell]{s1991}
      Sorell, T.\ \ {\it Scientism: Philosophy and the Infatuation with
      Science.}  Routledge.  New York.  1991.

   \bibitem[Vander Heeren(1912)Vander Heeren]{vdh1912}
      Vander Heeren, A.\ \ ``Septuagint Version.''\footnote{%
         \url{http://www.newadvent.org/cathen/13722a.htm}%
      }
      In {\it The Catholic Encyclopedia.}  1913 Edition.  Robert Appleton
      Company.  New York.  1912.

   \end{small}

\end{thebibliography}

\input{fdl-1.3}

\end{document}

