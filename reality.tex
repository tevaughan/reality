
\newcommand{\doctitle}{How Things Really Are, Actually and Potentially}

\documentclass[twocolumn]{article}

\usepackage{fancyhdr}
\usepackage{graphicx}
\usepackage{lastpage}
\usepackage{natbib}
\usepackage{times}
\usepackage{vmargin}

% hyperref must be last package.
\usepackage[colorlinks=true,citecolor=blue,hyperfootnotes=false]{hyperref}

% vmargin setup
\setpapersize{USletter}
\setmarginsrb%
{0.4in}%           left
{0.4in}%           top
{0.4in}%           right
{0.5in}%           bottom
{2\baselineskip}%  headheight
{2\baselineskip}%  headsep
{3\baselineskip}%  footheight
{4\baselineskip}%  footskip

% mydate macro
\newcommand{\mydate}{%
   \number\year\space%
   \ifcase\month\or%
      Jan\or\ Feb\or\ Mar\or\ Apr\or\ May\or\ Jun\or%
      Jul\or\ Aug\or\ Sep\or\ Oct\or\ Nov\or\ Dec
   \fi\space%
   \number\day%
}

% fancyhdr settings
\pagestyle{fancy}
\lhead{\sffamily\textbf{\doctitle}}
\chead{}
\rhead{\sffamily \thepage~of~\pageref{LastPage}}
\renewcommand{\headrulewidth}{1pt}
\renewcommand{\footrulewidth}{1pt}
\lfoot{%
   \scriptsize\sffamily
   \begin{minipage}{0.95\textwidth}
   Copyright~\copyright~2017~Thomas E. Vaughan.\ \ \
   PDF image generated on \mydate.\ \ \
   Source code and on-line PDF version at
      \url{https://github.com/tevaughan/reality}.\ \ \
   Permission is granted to copy, distribute and/or modify this document under
   the terms of the GNU Free Documentation License, Version 1.3 or any later
   version published by the Free Software Foundation; with no Invariant
   Sections, no Front-Cover Texts, and no Back-Cover Texts.  A copy of the
   license is included in the section entitled ``GNU Free Documentation
   License''.
   \end{minipage}%
}
\cfoot{}
\rfoot{%
   \begin{minipage}{0.05\textwidth}
   \begin{flushright}
   \includegraphics[width=0.85\textwidth]{logo}
   \end{flushright}
   \end{minipage}%
}

\begin{document}

\thispagestyle{fancy}

\section{Introduction}

There is a great war raging.  Western civilization is beseiged.  By ``Western
Civilization'' I mean the civilization that grew out of the marriage between
ancient Israel and classical Greece.  First, the old traditions became
intertwined in the writing of the {\it Septuagint}.\footnote{%
   The {\it Septuagint} is the first translation of the {\it Tanakh} (what
   Christians call the {\it Old Testament}).  The translation, into Greek, was
   made before the Christian era.  \cite{vdh1912} provides a detailed overview.
   \cite{r2005, r2007} comments on the marriage of Greek and Hebrew culture in
   the writing of the {\it Septuagint}.
}
Then the two became one in Christianity, which required both traditions in
order to express doctrine.  Christians converted the declining Roman empire to
the Faith, defended Europe against the onslaught of Muslim invaders,\footnote{%
   \citet[Chapter 4]{b1938} gives a concise overview of the circumstances
   leading up to the First Crusade.
}
and finally established a new civilization, among whose fruits are modern
science\footnote{%
   The terms ``science'' and ``scientific'' are by etymology misleading.  In
   Latin, the verb ``scire'' means ``to know,'' but what we call a ``scientific
   theory,'' is in many cases something that can never be known as a truth.
   There are, of course, truths in modern science; primarily, these are
   \emph{observational} truths: Every repeatable observation is a truth of
   modern science, a truth at least in the sense in which the observation was
   historically repeatable, and so a viable theory must explain that.  Further,
   a scientific theory that refers only to perceptible things might in
   principle be known as a truth, if it be a descriptive theory, such as the
   arrangement of the bones in the human body.  However, a scientific theory
   that refers to something imperceptible, like the electron, might some day be
   superseded by a new theory making no reference to the imperceptible thing.
   Its existence in reality is therefore not certain, and so the theory
   referring to it cannot be known as a truth.  There are many who would argue
   that the electron's existence is certain, but arguments for scientific
   realism have problems. \cite{c2016} gives an overview of scientific realism
   and its problems.%
}
and laws respecting the rights of man.  Western civilization has many enemies.
Among them are any who would deny a key Hebrew idea, such as that man is made
in the image of the Creator, and any who would deny a key Greek idea, such as
that a sentence cannot be both true and false at the same time and in the same
sense.

One enemy of Western civilization is a particular, modern form of atheism.  New
Atheists,\footnote{%
   ``New Atheism'' is a term widely used to refer to ideas popularized by
   Richard Dawkins, Sam Harris, Daniel Dennett, and Christopher Hitchens in the
   first decade of the 21st Century.  \cite{f2008} devotes a whole book to the
   refutation of the philosophical errors of the New Atheists.
}
such as Richard Dawkins, deny the Aristotelian metaphysical\footnote{%
   The word, ``metaphysics,'' originates from the order of the books in the
   traditional list of Aristotle's writings.  His writing on the distinction
   between act and potency came after (meta) his writing on nature (physics).
   These days, any theory about what lies at the root of being or of change is
   called ``a metaphysical theory.''%
}
basis for understanding reality while claiming for themselves the mantle of
science, much as the Soviet Communists did during the Cold War.  The present
talk is motivated by my interest in combating erroneous views about modern
science.  In the context of the great war and my general motivation, I point to
the philosophical conflict concerning how things really are.

\subsection{Ancient Israel and Creation}

The story of the conflict begins in ancient Israel.  The relevant idea from
{\it Genesis} is that things have not always existed.  God, Who is not a thing,
initially created things.  A new thing, such as a pot, might come into
existence by transformation from a pre-existing thing, such as a lump of clay.
However, according God's revelation to the Israelites, there was in the past a
beginning when the first things were created from nothing (ex nihilo), not by
transformation.  The fundamental distinction made by the ancient Israelites is
between God and what God creates.

\subsection{Classical Greece and Transformation}

Another thread of the story begins in classical Greece.  The early philosophers
struggled to understand how things change, and their tradition culminated in
the writing of Aristotle.  Aristotle explained that for a thing to change is
for what had existed only potentially in the thing to begin existing actually.
The fundamental distinction is between actuality and potentiality.  This
distinction is the root of what has come to be called ``Aristotelian
metaphysics.''

\subsection{Synthesis in Christianity}

When, after having been lost for a time, Aristotle was reintroduced into the
West by Saint Thomas Aquinas, he perfected the metaphysical theory so that it
could explain not only change by transformation but also creation ex nihilo.
Aquinas showed how, even if there had been no beginning in time, there would
still be creation ex nihilo at every moment in time.

\subsection{Misconceptions About Modern Science}

Although Aristotle's metaphysical theory naturally led to the emergence of
modern science in the West, Aristotle's opponents in the present conflict
ironically mount their attack under the banner of science.  The self-proclaimed
advocate\footnote{%
   Bill Nye, for example, is one of the most famous advocates of science.  Yet
   in advocating not merely science but scientism, he makes a philosophical
   error.  Bishop Robert Barron appears in a short movie in which he talks
   about Bill Nye's scientism:
   \url{https://www.youtube.com/watch?v=SH_Njsa0zVQ}.%
}
of science typically holds an erroneous view, \emph{scientism},\footnote{%
   Arguably the initial, definitive use of the word, ``scientism,'' is given by
   \cite{s1991}.  My definition is essentially the one given by \cite{h2011}.
   There are many arguments against scientism.  See, for example, Thomas
   Nagel's argument from the problem of qualia, or subjective experience.
   Nagel's view is summarized here:
   \url{http://www.nybooks.com/articles/2017/06/08/how-to-imagine-consciousness},
   in his response to a comment by a professor Black.  Note that Nagel is an
   atheist, though he is out of favor with the typical modern atheist, whose
   materialism takes the form of scientism.  \cite{f2014}, in the introductory
   chapter of his book, summarizes all of the main arguments against
   scientism.%
}
according to which everything that exists is describable by modern science. He
also typically holds a dubious view, \emph{scientific realism},\footnote{%
   \cite{c2016}.
}
according to which an imperceptible thing proposed as part of a scientific
theory is certain to exist.  We shall see in what follows how Aristotelian
metaphysics provides arguments at least against scientism, and I shall argue
also against scientific realism.  First, however, we must explore the key
Aristotelian distinction, which allowed Aquinas to form a synthesis of the
Judeo-Christian idea of creation and the Greek explanation of change.

\section{Act and Potency}

The earliest philosophers in classical Greece struggled with understanding how
change can happen.  The basic problem is that, along any continuum of change,
there is a discontinuity between any two distinct points on the continuum.

\subsection{Xeno's Paradox}

For example, if an ordinary object move from Point~A to Point~B, there is not a
discontinuous jump from A to B.  However, in considering any two points between
A and B, no matter how close together those two points might be, one would
still have a jump from the first to the second, even if one considered still
smaller separations forever.  This is one formulation of Xeno's paradox.

Xeno concluded that continuous movement is an illusion because even with an
infinite number of intermediate steps there would still be discontinuous jumps.

\subsection{The General Eleatic Problem}

\subsection{Heraclitus}

\subsection{The General Theory}

\subsubsection{Example: Silly Putty}

\subsection{Causal Powers and Laws of Nature}

\section{Efficient and Final Causality}

\section{Formal and Material Causality}

\section{Essence and Existence}

\bibliographystyle{plainnat}

\begin{thebibliography}{}

   \begin{small}

   \bibitem[Belloc(1938)Belloc]{b1938}
      Belloc, H.\ \ {\it The Great Heresies.}  Sheed and Ward.  London.  1938.

   \bibitem[Chakravartty(2016)Chakravartty]{c2016}
      Chakravartty, A.\ \ ``Scientific Realism.''\footnote{%
         \url{https://plato.stanford.edu/archives/win2016/entries/scientific-realism}%
      }
      In {\it The Stanford Encyclopedia of Philosophy.}  Winter 2016 edition.
      Metaphysics Research Lab.  Stanford University.  2016.

   \bibitem[Feser(2008)Feser]{f2008}
      Feser, E.\ \ {\it The Last Superstition: A Refutation of the New
      Atheism.}  St.~Augustine's Press.  2008.

   \bibitem[Feser(2014)Feser]{f2014}
      Feser, E.\ \ {\it Scholastic Metaphysics: A Contemporary Introduction.}
      editiones scholasticae.  Heusenstamm.  2014.

   \bibitem[Hutchinson(2011)Hutchinson]{h2011}
      Hutchinson, I.\ \ {\it Monopolizing Knowledge: A Scientist Refutes
      Religion-Denying, Reason-Destroying Scientism.}\footnote{%
         \url{http://monopolizingknowledge.net/contents.html}.%
      }
      Fias Publishing.  Belmont, MA.  2011.

   \bibitem[Ratzinger(2005)Ratzinger]{r2005}
      Ratzinger, J.\ \ {\it Truth and Tolerance: Christian Belief and World
      Religions.}  Ignatius Press.  San Francisco.  2004.

   \bibitem[Ratzinger(2007)Ratzinger]{r2007}
      Ratzinger, J.\ \ {\it Jesus of Nazareth.}  Doubleday.  New York.  2007.

   \bibitem[Sorell(1991)Sorell]{s1991}
      Sorell, T.\ \ {\it Scientism: Philosophy and the Infatuation with
      Science.}  Routledge.  New York.  1991.

   \bibitem[Vander Heeren(1912)Vander Heeren]{vdh1912}
      Vander Heeren, A.\ \ ``Septuagint Version.''\footnote{%
         \url{http://www.newadvent.org/cathen/13722a.htm}.%
      }
      In {\it The Catholic Encyclopedia.}  1913 edition.  Robert Appleton
      Company.  New York.  1912.

   \end{small}

\end{thebibliography}

\newpage

\input{fdl-1.3}

\end{document}

