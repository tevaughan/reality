
\documentclass[twocolumn]{article}

\usepackage{natbib}
\usepackage{times}
\usepackage{vmargin}

% hyperref must be last package.
\usepackage[colorlinks=true,citecolor=blue,hyperfootnotes=false]{hyperref}

% vmargin setup
\setpapersize{USletter}
\setmarginsrb%
{0.5in}%           left
{0.5in}%           top
{0.5in}%           right
{0.5in}%           bottom
{2\baselineskip}%  headheight
{2\baselineskip}%  headsep
{3\baselineskip}%  footheight
{4\baselineskip}%  footskip

\title{How Things Really Are\\{\Large Actually and Potentially}}
\author{Thomas E. Vaughan}

\begin{document}

\maketitle

\section{Introduction}

There is a great war raging.  Western civilization is under seige.  By
``Western Civilization'' I mean the civilization that grew out of the marriage
between ancient Israel and classical Greece.  First, the old traditions became
intertwined in the writing of the {\it Septuagint}.\footnote{%
   The {\it Septuagint} is the first translation of the Hebrew {\it Old
   Testament}.  The translation, into Greek, was made before the Christian era.
   See the article (\url{http://www.newadvent.org/cathen/13722a.htm}) on the
   {\it Septuagint} at New Advent.  See also Pope Benedict~XVI's comments on
   the marriage of Greek and Hebrew culture in each of \cite{r2005} and
   \cite{r2007}.
}
Then the two became one in Christianity, which required both traditions in
order to express doctrine. Christians converted the declining Roman empire to
the Faith, defended Europe against the onslaught of Muslim invaders,\footnote{%
   See Chapter~4 of \cite{b1938} for a brief overview of the circumstances
   leading up to the First Crusade.
}
and finally established a new civilization, among whose fruits are modern
science\footnote{%
   The terms ``science'' and ``scientific'' are by etymology misleading.  In
   Latin, the verb ``scire'' means ``to know,'' but what we call a ``scientific
   theory,'' is in many cases something that can never be known as a truth.
   There are, of course, truths in modern science; primarily, these are
   \emph{observational} truths: Every repeatable observation is a truth of
   modern science, a truth at least in the weak sense in which the observation
   is truly repeatable, and so any theory must explain that.  Further, a
   scientific theory that refers only to perceptible things might in principle
   be known as a truth, if it be a descriptive theory, such as the arrangement
   of the bones in the human body.  However, a scientific theory that refers to
   something imperceptible, like the electron for example, might some day be
   superseded by a new theory making no reference to the imperceptible thing.
   Its existence in reality is therefore not certain, and so the theory
   referring to it cannot be known as a truth.  There are many who would argue
   that the electron's existence is certain, but arguments for scientific
   realism have problems. See
   \url{https://plato.stanford.edu/entries/scientific-realism}.%
}
and laws respecting the rights of man.  One enemy of Western civilization is a
modern form of atheism.  The New Atheists\footnote{%
   ``New Atheism'' is a term widely used to refer to ideas popularized by
   Richard Dawkins, Sam Harris, Daniel Dennett, and Christopher Hitchens in the
   first decade of the 21st Century.  See \cite{f2008}.
}
deny the Aristotelian metaphysical basis for understanding reality while
claiming for themselves the mantle of science, much as the Soviet Communists
did during the Cold War.  The present talk is motivated by my interest in
combating erroneous views about modern science.  Now that I have introduced the
context of the great war and my general motivation, I point to the
philosophical conflict concerning how things really are.

The story of the conflict begins in ancient Israel.  The relevant idea from
{\it Genesis} is that things have not always existed.  God, Who is not a thing,
initially created things.  A new thing, such as a pot, might come into
existence by transformation from a pre-existing thing, such as a lump of clay.
However, according God's revelation to the Israelites, there was in the past a
beginning when the first things were created from nothing (ex nihilo), not by
transformation.  The fundamental distinction made by the ancient Israelites is
between God and what God creates.

Another thread of the story begins in classical Greece.  The early philosophers
struggled to understand how things change, and their tradition culminated in
the writing of Aristotle.  Aristotle explained that for a thing to change is
for what had existed only potentially in the thing to begin existing actually.
The fundamental distinction is between actuality and potentiality.  This
distinction is the root of what has come to be called ``Aristotelian
metaphysics.''\footnote{%
   The word, ``metaphysics,'' originates from the accidental fact that in the
   traditional list of Aristotle's writings, his writing on the distinction
   between act and potency came after (meta) his writing on nature (physics).
   These days, any theory about what lies at the root of being or becoming is
   called ``a metaphysical theory.''%
}

When, after having been lost for a time, Aristotle was reintroduced into the
West by Saint Thomas Aquinas, he perfected the metaphysical theory so that it
could explain not only change by transformation but also creation ex nihilo.
Aquinas showed how, even if there had been no beginning in time, there would
still be creation ex nihilo at every moment in time!  Although Aristotle's
metaphysical theory naturally led to the emergence of modern science in the
West, Aristotle's opponents in the present conflict ironically mount their
attack under the banner of science.  The self-proclaimed ``science advocate''
typically proposes a clear error: \emph{scientism},\footnote{%
   There are many arguments against scientism.  See, for example, Thomas
   Nagel's argument from the problem of qualia, or subjective experience. It is
   summarized here:
   \url{http://www.nybooks.com/articles/2017/06/08/how-to-imagine-consciousness}.
   Note that Nagel is an atheist, though he is out of favor with the typical
   modern atheist, whose materialism takes the form of scientism.%
}
according to which everything that exists is describable by modern science. He
also typically proposes a dubious position: \emph{scientific
realism},\footnote{%
   As in a previous note, see again
   \url{https://plato.stanford.edu/entries/scientific-realism}.
}
according to which an imperceptible thing proposed as part of a scientific
theory is certain to exist.  We shall see in what follows how Aristotelian
metaphysics provides powerful arguments against these errors.  First, however,
we must explore the key Aristotelian distinction, which allowed Aquinas to form
a synthesis of the Judeo-Christian idea of creation and the Greek explanation
of change.

\section{Act and Potency}

\subsection{The General Theory}

\subsubsection{Example: Silly Putty}

\subsection{Causal Powers and Laws of Nature}

\section{Efficient and Final Causality}

\section{Formal and Material Causality}

\section{Essence and Existence}

\bibliographystyle{plainnat}

\begin{thebibliography}{}

   \bibitem[Belloc(1938)Belloc]{b1938}
      Belloc, H.\ \ {\it The Great Heresies.}  Sheed and Ward.  London.  1938.

   \bibitem[Feser(2008)Feser]{f2008}
      Feser, E.\ \ {\it The Last Superstition: A Refutation of the New
      Atheism.}  St.~Augustine's Press.  2008.

   \bibitem[Feser(2014)Feser]{f2014}
      Feser, E.\ \ {\it Scholastic Metaphysics: A Contemporary Introduction.}
      editiones scholasticae.  Heusenstamm.  2014.

   \bibitem[Ratzinger(2005)Ratzinger]{r2005}
      Ratzinger, J.\ \ {\it Truth and Tolerance: Christian Belief and World
      Religions.}  Ignatius Press.  San Francisco.  2004.

   \bibitem[Ratzinger(2007)Ratzinger]{r2007}
      Ratzinger, J.\ \ {\it Jesus of Nazareth.}  Doubleday.  New York.  2007.

\end{thebibliography}

\end{document}

